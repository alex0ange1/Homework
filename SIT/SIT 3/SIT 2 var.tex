\documentclass{article}
\usepackage[T2A]{fontenc}
\usepackage[utf8]{inputenc}
\usepackage[english, russian]{babel}
\usepackage{amsthm}
\usepackage{amsmath}
\usepackage{amssymb}
\usepackage{amsfonts}
\usepackage{mathrsfs}
\usepackage[12pt]{extsizes}
\usepackage{fancyvrb}
\usepackage{indentfirst}
\usepackage[
  left=2cm, right=2cm, top=2cm, bottom=2cm, headsep=0.2cm, footskip=0.6cm, bindingoffset=0cm
]{geometry}
\usepackage[english,russian]{babel}

\begin{document}
\section*{Вариант 2}

Учитывая введенные переменные, и что в рассматриваемой постановке $\psi = o(1)$, $\upsilon_m/\omega_m = O(1)$, в нулевом приближении по $\psi$ и $\lambda$ получим линеаризованную задачу гидроупругости пульсирующего слоя жидкости в канале в виде уравнений динамики слоя жидкости

\begin{equation}
    \Re\frac{\partial\it U_{\xi0}}{\partial\tau} = -\frac{\partial P_0}{\partial\xi} + \frac{\partial^2 \it U_{\xi0}}{\partial\zeta^2}, \;\;\;\;
    \frac{\partial P_0}{\partial\zeta}=0,\;\;\;\;   \frac{\partial\it U_{\xi0}}{\partial\xi} + \frac{\partial \it U_{\zeta0}}{\partial\zeta}=0,
\end{equation}
с граничными условиями

\begin{equation}
    \it U_{\xi0} = \text{0}, \; \it U_{\zeta0} = \partial\text{W}/\partial\tau \text{ при } \zeta = \text{1}; \;\;\; U_{\xi0} = \text{0}, \;U_{\zeta0} = \text{0} \text{ при } \zeta = \text{0};
\end{equation}
\begin{equation}
    P = P^+(\tau) \text{ при } \xi = 1; \;\;\; P = 0 \text{ при } \xi = -1;
\end{equation}
и уравнения движения пластинки-стенки канала

\begin{equation}
    \frac{Dw_m}{\ell^4} \frac{\partial^4W}{\partial\xi^4} + \rho_0h_0\omega^2w_m\frac{\partial^2W}{\partial\tau^2} = p_0+\frac{w_m\rho\nu\omega}{\delta_0\psi^2}P
\end{equation}
с граничными условиями

\begin{equation}
    W = \partial^2W/\partial\xi^2 = 0 \text{ при }\xi = \pm1
\end{equation}

\end{document}

